\section{Data Structures}\label{sec:data-structures}

\subsection{NodeInfoRes}\label{struct:NodeInfoRes}
Response containing information about a node's capabilities.


\begin{figure}[H]
\centering
\begin{bytefield}[
bitwidth=\widthof{liquid\_hash },
leftcurly=., leftcurlyspace=2pt]{8}
\begin{leftwordgroup} {0}
 \bitheader[bitformatting=\large]{0-7} \\
\bitbox{1}{tel\_cnt} & \bitbox{1}{par\_cnt} & 
\bitbox{4}{firmware\_hash} &
& &  \bitbox{2}{liquid\_hash  }
\end{leftwordgroup}\\

\begin{leftwordgroup} {1}
\bitbox{2}{liquid\_hash} \bitbox[lrt]{6}{  }
\end{leftwordgroup}\\

\begin{leftwordgroup} {2}
\bitbox[lr]{8}{  }
\end{leftwordgroup}\\
\begin{leftwordgroup} {3}
\bitbox[lr]{8}{  }
\end{leftwordgroup}\\
\begin{leftwordgroup} {4}
\bitbox[lr]{8}{ device\_name }
\end{leftwordgroup}\\
\begin{leftwordgroup} {5}
\bitbox[lr]{8}{  }
\end{leftwordgroup}\\
\begin{leftwordgroup} {6}
\bitbox[lr]{8}{  }
\end{leftwordgroup}\\
\begin{leftwordgroup} {7}
\bitbox[lrb]{7}{  }
\bitbox[t]{1}{}
\end{leftwordgroup}\\

\end{bytefield}

\vspace{2mm}


\caption{NodeInfoRes byte layout (8 bytes per row)}
\end{figure}

\textbf{Field Descriptions:}

\begin{center}
\begin{tabular}{l c l}
\toprule
Field & Bytes & Description \\
\midrule
tel\_cnt & 1 & Number of telemetryValues on this node \\
par\_cnt & 1 & Number of parameters on this node \\
firmware\_hash & 4 & Hash of the firmware version \\
liquid\_hash & 4 & Hash of the LiquidCan protocol version \\
device\_name & 53 & Human-readable device name \\
\midrule
\textbf{Total} & \textbf{63} & \\
\bottomrule
\end{tabular}
\end{center}

% TODO: Describe the purpose and usage of NodeInfoRes

\begin{lstlisting}[caption={NodeInfoRes struct}]
typedef struct __attribute__((packed)) {
    uint8_t var_count;
    uint8_t par_count;
    uint32_t firmware_hash;
    uint32_t liquid_hash;
    char device_name[53];
} node_info_res_t;
\end{lstlisting}

\subsection{Status}\label{struct:Status}
General status message with text information.

\begin{center}
\begin{tabular}{l c l}
\toprule
Field & Bytes & Description \\
\midrule
msg & 63 & Status message text \\
\bottomrule
\end{tabular}
\end{center}

% TODO: Describe when status messages are used

\begin{lstlisting}[caption={Status struct}]
typedef struct __attribute__((packed)) {
    char msg[63];
} status_t;
\end{lstlisting}

\subsection{FieldRegistration}\label{struct:FieldRegistration}
Registration information for a telemetryValue or parameter field.
The DataType here refers to the DataType Enum value (see \ref{subsec:DataType}).

\begin{center}
\begin{tabular}{l c l}
\toprule
Field & Bytes & Description \\
\midrule
field\_id & 1 & Unique identifier for this field \\
field\_type & 1 & Data type (DataType enum) \\
field\_name & 61 & Human-readable field name \\
\midrule
\textbf{Total} & \textbf{63} & \\
\bottomrule
\end{tabular}
\end{center}


\begin{lstlisting}[caption={FieldRegistration struct}]
typedef struct __attribute__((packed)) {
    uint8_t field_id;
    uint8_t field_type; 
    char field_name[61];
} field_registration_t;
\end{lstlisting}

\subsection{TelemetryGroupDefinition}\label{struct:TelemetryGroupDefinition}
Defines a group of related telemetryValues for efficient batch updates.
\begin{center}
\begin{tabular}{l c l}
\toprule
Field & Bytes & Description \\
\midrule
group\_id & 1 & Unique identifier for this group \\
field\_ids & 62 & Array of field IDs in this group \\
\midrule
\textbf{Total} & \textbf{63} & \\
\bottomrule
\end{tabular}
\end{center}

\begin{lstlisting}[caption={TelemetryGroupDefinition struct}]
typedef struct __attribute__((packed)) {
    uint8_t group_id;
    uint8_t field_ids[62];
} telemetry_group_definition_t;
\end{lstlisting}

\subsection{TelemetryGroupUpdate}\label{struct:TelemetryGroupUpdate}
Updates all telemetryValues from a previously defined group.

\begin{center}
\begin{tabular}{l c l}
\toprule
Field & Bytes & Description \\
\midrule
group\_id & 1 & Group identifier \\
values & 62 & Packed values of all telemetry values in the group \\
\midrule
\textbf{Total} & \textbf{63} & \\
\bottomrule
\end{tabular}
\end{center}

\textbf{Note:} The values are packed in the same order as announced in the TelemetryGroupDefinition.
 
% TODO: Describe value packing rules and alignment

\begin{lstlisting}[caption={TelemetryGroupUpdate struct}]
typedef struct __attribute__((packed)) {
    uint8_t group_id;
    uint8_t values[62];
} telemetry_group_update_t;
\end{lstlisting}

\subsection{HeartBeat}\label{struct:HeartBeat}
Heartbeat message with counter.

\begin{center}
\begin{tabular}{l c l}
\toprule
Field & Bytes & Description \\
\midrule
counter & 4 & Incrementing counter value \\
\bottomrule
\end{tabular}
\end{center}

% TODO: Describe heartbeat interval and timeout behavior

\begin{lstlisting}[caption={HeartBeat struct}]
typedef struct __attribute__((packed)) {
    uint32_t counter;
} heartbeat_t;
\end{lstlisting}

\subsection{ParameterSetReq}\label{struct:ParameterSetReq}
Request to set a parameter value.

\begin{center}
\begin{tabular}{l c l}
\toprule
Field & Bytes & Description \\
\midrule
parameter\_id & 1 & Parameter identifier \\
value & 61 & New value (type depends on parameter) \\
\midrule
\textbf{Total} & \textbf{62} & \\
\bottomrule
\end{tabular}
\end{center}

\begin{lstlisting}[caption={ParameterSetReq struct}]
typedef struct __attribute__((packed)) {
    uint8_t parameter_id;
    uint8_t value[61];
} parameter_set_req_t;
\end{lstlisting}

\subsection{ParameterSetStatus}\label{subsec:ParameterSetStatus}
Status codes for parameter set operations:
\begin{center}
\begin{tabular}{l l l}
\toprule
Enum Value & Status Name & Description \\
\midrule
0 & \texttt{Success} & Parameter was successfully set \\
1 & \texttt{InvalidParameterID} & The parameter ID does not exist \\
2 & \texttt{ParameterLocked} & The parameter is locked and cannot be modified \\
3 & \texttt{NodeToNodeModification} & The parameter was modified by another node\\
\bottomrule
\end{tabular}
\end{center}

\subsection{ParameterSetConfirmation}\label{struct:ParameterSetConfirmation}
Response to a parameter set request.

\begin{center}
\begin{tabular}{l c l}
\toprule
Field & Bytes & Description \\
\midrule
parameter\_id & 1 & Parameter identifier \\
status & 1 & Status code (ParameterSetStatus enum) \\
value & 61 & Confirmed value after set operation \\
\midrule
\textbf{Total} & \textbf{63} & \\
\bottomrule
\end{tabular}
\end{center}


\begin{lstlisting}[caption={ParameterSetConfirmation struct}]
typedef struct __attribute__((packed)) {
    uint8_t parameter_id;
    uint8_t status;
    uint8_t value[61];
} parameter_set_confirmation_t;
\end{lstlisting}

\subsection{FieldGetReq}\label{struct:FieldGetReq}
Request to retrieve a field value
\begin{center}
\begin{tabular}{l c l}
\toprule
Field & Bits/Bytes & Description \\
\midrule
field\_id & 1 Byte & Field identifier \\
\bottomrule
\end{tabular}
\end{center}

% TODO: Describe field retrieval process

\begin{lstlisting}[caption={FieldGetReq struct}]
typedef struct __attribute__((packed)) {
    uint8_t field_id;
} field_get_req_t;
\end{lstlisting}

\subsection{FieldGetRes}\label{struct:FieldGetRes}
Response with requested field value.

\begin{center}
\begin{tabular}{l c l}
\toprule
Field & Bytes & Description \\
\midrule
field\_id & 1 & Field identifier \\
value & 62 & Field value \\
\midrule
\textbf{Total} & \textbf{63} & \\
\bottomrule
\end{tabular}
\end{center}

% TODO: Describe response format

\begin{lstlisting}[caption={FieldGetRes struct}]
typedef struct __attribute__((packed)) {
    uint8_t field_id;
    uint8_t value[62];
} field_get_res_t;
\end{lstlisting}


\subsection{FieldIDLookupReq}\label{struct:FieldIDLookupReq}
Requsts the fieldID of a field matching a field name

\begin{center}
\begin{tabular}{l c l}
\toprule
Field & Bytes & Description \\
\midrule
field\_name & 61 & Field Name \\
\midrule
\textbf{Total} & \textbf{61} & \\
\bottomrule
\end{tabular}
\end{center}

% TODO: Describe response format

\begin{lstlisting}[caption={FieldIDLookupReq struct}]
typedef struct __attribute__((packed)) {
    uint8_t field_name[61];
} field_id_lookup_req_t;
\end{lstlisting}

\subsection{FieldIDLookupRes}\label{struct:FieldIDLookupRes}
Response with requested field id and the datatype of the field.
\\
\begin{center}
\begin{tabular}{l c l}
\toprule
Field & Bytes & Description \\
\midrule
field\_id & 1 & Field ID \\
field\_type & 1 & Field Datatype\\
\midrule
\textbf{Total} & \textbf{2} & \\
\bottomrule
\end{tabular}
\end{center}

% TODO: Describe response format

\begin{lstlisting}[caption={FieldIDLookupRes struct}]
typedef struct __attribute__((packed)) {
    uint8_t fieldID;
    uint8_t field_type;
} field_id_lookup_res_t;
\end{lstlisting}

\subsection{ParameterSetLock}\label{struct:ParameterSetLock}
Locks a parameter to prevent changes.

\begin{center}
\begin{tabular}{l c l}
\toprule
Field & Bytes & Description \\
\midrule
parameter\_id & 1 & Parameter identifier to lock \\
parameter\_lock & 1 & Lock status(0=unlocked, 1=locked) \\
\midrule
\textbf{Total} & \textbf{2} & \\
\bottomrule
\end{tabular}
\end{center}

% TODO: Describe parameter locking mechanism and use cases

\begin{lstlisting}[caption={ParameterSetLock struct}]
typedef struct __attribute__((packed)) {
    uint8_t parameter_id;
    uint8_t lock_status;
} parameter_set_lock_t;
\end{lstlisting}


\subsection{DataType}\label{subsec:DataType}
The protocol supports the following data types:
\begin{center}
\begin{tabular}{l l l}
\toprule
Enum Values & Type Name & Description \\
\midrule
0 & \texttt{Float32} & 32-bit floating point \\
1 & \texttt{Int32} & 32-bit signed integer \\
2 & \texttt{Int16} & 16-bit signed integer \\
3 & \texttt{Int8} & 8-bit signed integer \\
4 & \texttt{UInt32} & 32-bit unsigned integer \\
5 & \texttt{UInt16} & 16-bit unsigned integer \\
6 & \texttt{UInt8} & 8-bit unsigned integer \\
7 & \texttt{Boolean} & 1-bit boolean value \\

\bottomrule
\end{tabular}
\end{center}
