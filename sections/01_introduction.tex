\section{Purpose and Scope}
% TODO: Describe the purpose of the LiquidCan protocol
% Nodes involved:
% Expected CAN bitrate(s):
% Error handling policy:
% Versioning scheme:
% Interoperability constraints:
The purpose of the LiquidCan protocol is to serve at the heart of all future Liquids Projects at the TU Wien Space Team.
Building on the CAN FD Standard, it defines the way our client devices (such as ECUs) communicate with the central server and with each other.
It is designed to be as simple and extensible as possible. Care has been taken to minimize the amount of common type definitions between the server and the nodes.

\section{Notation and Conventions}\label{sec:notation}
\begin{itemize}
    \item This protocol uses the CAN-FD extension
    \item All fields are little endian.
    \item Payload length: 64 Byte CAN FD.
\end{itemize}

\subsection{Common Terms}\label{subsec:common-terms}
\begin{center}
\begin{tabular}{l l}
\toprule
Term & Description \\
\midrule
(CAN) Client & A CAN client is a device which is connected to the bus\\ 
Node & Every client which is not the server\\ 
Telemetry & A telemetry is a non-externally modifiable value which gets periodically sent to the server\\
Parameters & Parameters can be externally modified\\
Field & An encompassing term for telemetries and parameters\\ 
ECU & A commonly used embedded CAN device at the TU Wien Space Team

\end{tabular}
\end{center}
