\section{Field Registration \& Management}\label{sec:field-registration}
Fields are the heart of the protocol. The term Field serves as a general term for both telemetries and parameters. 
Telemetries are periodically sent to the server and non-modifiable. They are meant to represent sensor data or other 
information which should be periodically logged.
Parameters can be externally modified and locked. These are meant for configuration telemetries, modifiable by either the server or other nodes.

\begin{figure}[H]
\centering
\begin{tikzpicture}[
    node distance=10mm,
    every node/.style={font=\small},
    box/.style={rectangle, draw, rounded corners, minimum width=3.8cm, minimum height=8mm, align=center, fill=blue!10},
    serverbox/.style={rectangle, draw, rounded corners, minimum width=3.8cm, minimum height=8mm, align=center, fill=orange!10},
    arrow/.style={-{Stealth[]}, thick},
    dasharrow/.style={-{Stealth[]}, thick, dashed}
]
    % Node column
    \node[box] (start) {Node Powers On};
    \node[box, below=of start] (sendinfo) {Send\\ Node Infos};
    \node[box, below=of sendinfo] (sendreg) {Send Field\\Registrations\\(for each field)};
    \node[box, below=of sendreg] (sendgroup) {Send Field Group\\Definitions};
    \node[box, below=of sendgroup] (ready) {Node Ready};
    \node[box, below=14mm of ready] (periodic) {Periodic\\Field Group Updates};
    
    % Server column
    \node[serverbox, right=47mm of sendinfo] (rcvinfo) {Register Node\\Store Metadata};
    \node[serverbox, right=47mm of sendreg] (rcvreg) {Store Field\\Metadata};
    \node[serverbox, right=47mm of sendgroup] (rcvgroup) {Store Group\\Definitions};
    \node[serverbox, right=47mm of periodic] (process) {Process Updates\\Store Values};
    
    % Labels
    \node[above=3mm of start, font=\bfseries] {Client Node};
    \node[above=3mm of rcvinfo, font=\bfseries] {Server};
    
    % Arrows - registration phase
    \draw[arrow] (start) -- (sendinfo);
    \draw[arrow] (sendinfo) -- (sendreg);
    \draw[arrow] (sendreg) -- (sendgroup);
    \draw[arrow] (sendgroup) -- (ready);
    \draw[arrow] (ready) -- (periodic);
    
    % Message arrows
    \draw[arrow] (sendinfo.east) -- (rcvinfo.west) node[midway, above, font=\footnotesize] {\texttt{node\_info\_res}};
    \draw[arrow] (sendreg.east) -- (rcvreg.west) node[midway, above, font=\footnotesize] {\texttt{field\_registration}};
    \draw[arrow] (sendgroup.east) -- (rcvgroup.west) node[midway, above, font=\footnotesize] {\texttt{field\_group\_definition}};
    \draw[arrow] (periodic.east) -- (process.west) node[midway, above, font=\footnotesize] {\texttt{field\_group\_update}};
    
    % Phases - asymmetric padding (more on left)
    \node[draw, dashed, fit=(sendinfo) (sendreg) (sendgroup), inner xsep=8mm, inner ysep=8mm, xshift=-4mm] (regphase) {};
    \node[draw, dashed, fit=(periodic), inner xsep=8mm, inner ysep=6mm, xshift=-4mm,yshift=2mm] (opphase) {};
    
    % Phase labels - positioned outside the boxes
    \node[anchor=north west, font=\small\itshape] at (regphase.north west) {Registration Phase};
    \node[anchor=north west, font=\small\itshape] at (opphase.north west) {Normal Operation};
    
    % Loop arrow for periodic updates
    \draw[arrow] (periodic.south) -- ++(0,-8mm) -| ($(periodic.west)+(-16mm,0)$) -- (periodic.west);
    
\end{tikzpicture}
\caption{Field Registration and Update Flow}
\end{figure}

\subsection{Registration}\label{subsec:registration}
Telemetries and parameters are dynamically defined over the bus.
\paragraph{}
After registration was initialized, a node sends out \texttt{field\_registration} messages, one for each 
parameter/telemetry. The FieldRegistration includes a field ID, the datatype of the field, and 
a human-readable name. The name and field ID have to be unique per node. The first bit of paramter field IDs is 0 and the first bit of telemetry field IDs is 1. From this point on, the server knows which fields the node has.

\paragraph{}
Next, the node sends \texttt{field\_group\_definition} messages. Telemetries and parameters cannot be mixed together in one FieldGroupDefinition.
The node sends a group ID to identify the group and a list of field IDs. 
The order of the field IDs must be the same as the order of fields in future \texttt{field\_group\_updates}. The node must ensure that the values all of the fields defined can fit into a FieldGroupUpdate.


\subsection{Regular Operation}\label{subsec:regular-operation}
\subsubsection{Field Updates}\label{subsubsec:field-updates}
During regular operation, each node sends \texttt{field\_group\_update} messages at a defined interval. The order of values here must match the field id ordering of the corresponding \texttt{field\_group\_definition}.
The interval can vary between groups, allowing nodes to send fields at different intervals to, for example, reduce bus utilization.

\subsubsection{Parameter Setting}\label{subsubsec:parameter-setting}
Other bus members can send a \texttt{parameter\_set\_req} message, which includes the field ID of the parameter and the new value.
Once the node receives the request, it changes the internal parameter value and responds with a \texttt{parameter\_set\_res} message containing the 
new value. This should be the actual value read back from the parameter, not simply the value that was received in the request.


When a parameter is internally modified through some automated system, the updated value must be sent as a \texttt{parameter\_set\_res} message to the server.
\subsubsection{Parameter Locking}\label{subsubsec:parameter-locking}

A parameter can optionally be locked through a \texttt{parameter\_set\_lock\_req} message .
After a parameter has been locked, it cannot be modified by an external node.
A parameter can only be unlocked by the locking node or the server. To lock a parameter, a node sends a \texttt{parameter\_set\_lock\_req} with the fieldID and the locking status (0=unlocked, 1=locked). The recieving node responds with a \texttt{parameter\_set\_lock\_res}, confirming the sent fieldID and the locking status.
\paragraph{}
\subsubsection{Requesting Field Data}\label{subsubsec:requesting-field-data}
A field can be accessed through a \texttt{field\_get\_req} message, which contains the field ID.
Nodes respond with a \texttt{field\_get\_res} message, containing the field ID and the value of the field.

\subsubsection{Field name Lookup}\label{subsubsec:field-name-lookup}
The field name lookup covers the case where nodes need to access fields from other nodes. Since they don't recieve the \texttt{field\_registration} messages, they don't know the fieldIDs of the named fields they want to access. \texttt{field\_id\_lookup\_req} messages contain the remote field name. The Node responds with a \texttt{field\_id\_lookup\_res} message, containing a bit which states if the field was found and the fieldID.